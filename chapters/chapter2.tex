\section{Chapter 2}

\begin{definition}[\bfseries Function types]
  For a function $f:X\rightarrow Y$ (domain to codomain): \\
  \textbf{surjective} is at least one arrow per ele in codomain (codomain overmapped): $\forall y \in Y \exists x \in X \text{ s.t. } y = f(x)$\\
  \textbf{injective} is at most one in arrow per ele in codomain (codomain undermapped): $\forall x, x' \in X,f(x) = f(x')\implies x = x'$\\
  \textbf{bijective} is one to one: both surjective and injective

\end{definition}

\begin{definition}[\bfseries Law of Composition]
  A function of two variables:
  \[S \times S \rightarrow S\]

  Where $S \times S$ is the product of two sets (set of pairs).
\end{definition}

\begin{definition}[\bfseries Examples of laws of composition]
  Here's two examples:

  \begin{itemize}
      \item (ab) c = a(bc) (associative law)
      \item ab = ba (communative)
  \end{itemize}
\end{definition}

\begin{definition}[\bfseries Group]
  A group is a set G and a law of composition with the following properties:
  \begin{itemize}
    \item The law of composition is associative: $(ab)c = a(bc)$
    \item G contains an identity element 1 s.t. $1a = a$ and  $ a1 = a$ forall $a$ in G
    \item Every element a of G has an inverse, an element b s.t. $ab=1$ and  $ba = 1$
  \end{itemize}
  A group's order (number elemenets) is its cardinality.
\end{definition}
\begin{definition}[\bfseries Abelian Group]
  A group whose law of composition is commutative. That is to say, $ab = ba$.
  Also denoted a $\textbf{commutative group}$
\end{definition}

And now for some common groups\ldots

\begin{definition}[\bfseries $n\times n$ Common groups]
  General linear group: group of all invertible n by n matrices. Denoted $GL_{n}$ \\
  Special linear group: group of all n by n matrices with determinant 1. Denoted $SL_{n}$. \\
  Alternating group: set of even permutations. Denoted $A_{n}$ \\
\end{definition}

\begin{definition}[\bfseries permutation group]
  The group of permutations of the set of n indices is called the \textbf{symmetric group}.
  This includes all permutations of those n indices.

\end{definition}

\begin{definition}[\bfseries Subgroup]
  A subset H of a group G is a subgroup if it has the following properties:
  \begin{itemize}
    \item Closure: if a and b are in H, then ab is in H
    \item Identity: 1 is in H
    \item Inverses: If a is in H, then $a^{-1}$ is also in H
  \end{itemize}

  There are two \emph{trivial} subgroups for any group G. The subgroup
  containing every element in G, and the subgroup only containing the identity.
  Subgroups that are not \emph{trivial} are denoted \emph{proper}.

\end{definition}

\begin{definition}[\bfseries divides]
Given two integers $a$ and $b$, we say $a$ divides $b$ if there is an integer
$c$ such that $b=ac$. E.g. $a$ divides $b$ if $b$ is a multiple of $a$.
Equivalently, if $a$ is a factor of $b$.
\end{definition}

\begin{definition}[\bfseries Subgroup of Additive Group of Integers]
  A subset of a group $G$ with law of composition written additively is a subgroup
  if it has these properties:
  \begin{itemize}
      \item Closure: if $a$ and $b$ are in $S$ then $a + b$ is in $S$
      \item Identity: 0 is in $S$
      \item Inverses: If $a$ is in $S$ then $-a$ is in $S$
  \end{itemize}

  Define an additive subgroup notationally like this:
  \[\mathbb{Z}a = \{n \in \mathbb{Z} | n = ka \text{ for some } k \in \mathbb{Z}\}\]

  Also union of two groups is gcd, difference is lcm.
\end{definition}

\begin{definition}[\bfseries bracket notation]
  $<x>$ means smallest subgroup generated by single element $x$ in group G.
  Using multiplicative notation, this means:
  \[< x > = \{\ldots, x^{-1}, 1, x, x^{2}\}\]
\end{definition}

\begin{definition}[\bfseries order of element]
  The order of an element $n$ in a group $G$ is the smallest positive integer
  with the property $x^{n}=1$
\end{definition}

\begin{definition}[\bfseries homomorphism]
  Let $G$ and $G'$ be groups written with multiplicative notation. A
  homomorphism $\phi:G\rightarrow G'$ is a map from $G$ to $G'$ s.t. $\forall a,b\in G$:
  \[\phi (ab)=\phi (a) \phi (b)\]

  Trivial homomorphism maps every element in $G$ to the identity in $G'$.

  Note that this preserves both identity and inverse relations.

  The image of a homomorphism is the set elements mapped to in $G'$ from G
  THIS IS A SUBSET OF $G$. This is a subgroup of the range.
  \[\phi ( G ) = \text{im} \phi = \{x \in G' | \exists a \text{ s.t. }x = \phi (a)\}\]

  The kernel of a homomorphism is the set of elements of $G$ that map to
  $G'_{id}$. This is also a subgroup.
  \[\ker \phi = \{a \in G | \phi(a) = \text{id}\}\]

  The kernel is a normal subgroup of $G$.

  If the kernel is trivial, then this implies the mapping is injective because
  $a^{-1}b \in \ker\phi\implies \phi (a)=\phi (b)$. But we can't have this by
  defn b/c then the kernel is no longer trivial.

  Congruence relation is defined to be elements $a,b$ s.t. $\phi(a) = \phi(b)$

\end{definition}

\begin{definition}[\bfseries cosets]
  Let $H$ be a subgroup of a group $G$ and $a$ be an element of $G$. Then define:

  \[aH = \{g \in G | g = ah \forall h \in H\} \text{ (left coset) }\]
  \[Ha = \{g \in G | g = ha \forall h \in H\} \text{ (right coset) }\]

  The number of left cosets in a subgroup is called \textbf{index} of $H$ in $G$
  and denoted $[G:H]$. Let $G \subset H \subset K$ be subgroups of a group $G$.
  Then $[G:K] = [G:H][H:K]$

  If is normal subgroup, left and right cosets are equal.
\end{definition}

\begin{definition}[\bfseries conjugate]
  If a and g are elements of a group $G$, then the element $gag^{{-1}}$ is
  the conjugate of $a$ by $g$.
\end{definition}

\begin{definition}[\bfseries normal subgroup]
  A subgroup $N$ of a group $G$ is a normal subgroup if for every $a$ in $N$ and
  every $g$ in $G$, the conjugate $gag^{-1}$ is also in $N$.
\end{definition}

\begin{definition}[\bfseries center of group]
  The center of group $G$, denoted $Z$, is defined as:
  \[Z = \{z \in G | zx = xz \forall x \in G\}\]

  This is always normal (because commutivity)
\end{definition}

\begin{definition}[\bfseries isomorphism]
  An isomorphism $\phi: G \rightarrow G'$ between groups $G$ and $G'$ is a
  \textbf{bijective} group homomorphism. E.g. each element is only mapped to
  once. This means there is another isomorphism $\phi^{{-1}}: G' \rightarrow G$.
  We denote isomorphic to as $G \approx G'$.
\end{definition}

\begin{definition}[\bfseries Isomorphic Class]
  The groups isomorphic to a given group $G$ form what is called the isomorphism
  class of $G$.
\end{definition}

\begin{definition}[\bfseries Automorphism]
  An automorphism on G is defined to be a isomorphism $\phi: G \rightarrow G$.
\end{definition}

\begin{definition}[\bfseries Conjugation]

  Let $g$ be a fixed element of a group $G$. Conjugation by $g$ is the map
  $\phi(x) = g x g^{-1}$. This is (fairly obviously) an automorphism.

\end{definition}

\begin{definition}[\bfseries commuting in a group]
  The \textbf{commutator} $aba^{-1}b^{_{-1}}$ is an element associated with a pair $(a,b)$
  in a group. Two elements a and b of a group commute, eg $ab=ba$, iff
  \[aba^{-1}=b \leftarrow \rightarrow aba^{-1}b^{-1} = 1\]

\end{definition}

\begin{definition}[\bfseries partition]
  A partition $\Pi$ of a set $S$ is a subdivision of $S$ into nonoverlapping,
  nonempty subsets.
\end{definition}

\begin{definition}[\bfseries equivalence relation]
  An equivalence relation on a set S is a relation that holds between certain
  pairs of elements of $S$. We denote this $a \sim b$. Requirements are:
  \begin{itemize}
      \item \textbf{transitive}: if $a \sim b$ and $b \sim c$, then $a \sim c$
      \item \textbf{symmetric}: if $a \sim b$, then $b \sim a$
      \item \textbf{reflexive}: $\forall a, a \sim a$
  \end{itemize}

  With group homomorphisms, the equivalence relation is defined by:
  \[\phi(a) = \phi(b) \rightarrow a \equiv b\]

\end{definition}

\begin{definition}[\bfseries equivalence classes]

  An equivalent relation defines a partition (and vice versa). All the elements
  in one of the subsets in the partition are equivalent by $\sim$. These subsets
  are denoted \textbf{equivalence classes}. Bar is used to denote equivalence
  class. The set of equivalence classes from a set S is denoted $\bar{S}$. An
  equivalence class of element $b$ is denoted $\bar{b}$.

\end{definition}


\begin{definition}[\bfseries maps]

  For a map of sets $f:S \rightarrow T$, the inverse image, or $fibre$ is defined:
  \[f^{-1}(t) = \{s \in S | f(s) = t\}\]

  Non-empty fibres are equivalence classes for equivalence relation
  $a \sim b $ if $f(a) = f(b)$.

\end{definition}

\begin{definition}[\bfseries Counting Formula]
  Note that all left cosets $aH$ of a subgroup $H$ of a group $G$ have the same
  order, $|H|$, because the mapping is bijective (any element a has an inverse so
  you can undo whatever you did). We also know that these cosets partition $G$
  (each element in g times the id element in H means g is in some coset). As a
  result:
  \[|G| = |H|[G:H]\]
  Colloraries:
  \begin{itemize}
    \item
      Langrange's theorem states if $H$ is a subgroup of finite group $G$, then
      the order of $H$ divides the order of $G$. Another collorary is (noting that the
      kernel is a subgroup), $|G| = |\ker \phi| * | \ima \phi |$ Note that to prove
      this, use the fact $[G:\ker \phi] = | \ima \phi |$. This holds true because the
      LHS is the number of nonempty fibres in $G'$, and the RHS is the number of
      elements mapped to in $G'$.
    \item $|\ima \phi|$ also divides $G'$ b/c the image is a subgroup of $G'$
      (follows from lagrange).
  \end{itemize}
\end{definition}

\begin{definition}[\bfseries Congruence Relation]

  Given some subgroup $H$ of a group $G$, right and left congruence are defined
  as follows:
  \[a \equiv b \text{ if } b = ah \text{ for some } h \in H  (\text{left congruence})\]
  \[a \equiv b \text{ if } b = ha \text{ for some } h \in H  (\text{right congruence})\]

  These cosets both individually partition $G$.

\end{definition}

\begin{definition}[\bfseries congruence classes modulo $n$]

  Denoted $\mathbb{Z}/ \mathbb{Z}n$, $\mathbb{Z} / n \mathbb{Z}$, or
  $\mathbb{Z}/(n)$. Means set of congruence classes modulo $n: $congruence
  classes $\mathbb{Z}i$ for $i \in[0,n-1]$.

\end{definition}

\begin{definition}[\bfseries restricting homomorphism]

  Let $\psi:G \rightarrow G'$ be a homomorphism and let $H$ be a subgroup of $G$. We may
  restrict $\phi$ to $H$ to get the homomorphism:
  \[\phi|_{H}: H \rightarrow G'\]
  Same map $\phi$, smaller domain.

\end{definition}

\begin{definition}[\bfseries Correspondence Theorem]
  Let $\phi: G \rightarrow G'$ be a surjective homomorphism with kernel $K$. There is a
  bijective correspondence between subgroups of $G'$ and subgroups of $G$ that
  contain $K$. Then we have that that map is:
  \[\text{subgroup H of G that contains K} \rightarrow \text{ its image } \phi(H) \in G'\]
  \[\text{a subgroup H' of G'} \rightarrow \text{ its inverse image }\phi^{-1}(H') \in G\]

  If $H$ and $H'$ are corresponding subgrupos, then $H$ is normal in $G$ iff $H'$
  is normal in $G'$.

  If $H$ and $H'$ are corresponding subgroups, then $|H| = |H'||K|$
  (follows naturally from counting formula).

\end{definition}

\begin{definition}[\bfseries Product group]

  A product group is defined to be cartesian product of two groups with pairs.
  Multiplication is pairwise--$(a, b) \cdot (a',b')=(aa', bb')$. It's denoted $G \times G'$.
  Mapping from product group back to $G$ or $G'$ is called a projection.

\end{definition}

\begin{definition}[\bfseries Quotient Group]
  \[\bar{G} = G / N \text{ is the set of cosets of a normal subgroup} N
    \text{ in group} G\]

  A theorem for why quotient groups are useful:

  Let $N$ is a normal subgroup of a group $G$, and let $\bar{G}$ denote the set
  of cosets of $N$ in $G$. There is a law of composition of $\bar{G}$ that makes
  this set into a group s.t. the map $\phi:G \rightarrow \bar{G}$ defined by
  $\pi(a) \rightarrow \bar{a}$ is a surjective homomorphism whose kernel is $N$.

  $\pi$ is denoted \textbf{canonical map} from $G$ to $\bar{G}$.


\end{definition}

\begin{definition}[\bfseries Proper Subgroup]
  The subgroup is an actual subset; e.g. G and H (H is subset of G) are not the
  same.

\end{definition}

\begin{definition}[\bfseries Product Set]

  If $A$ and $B$ are subsets of a group $G$, then $AB$ denotes the set of products:
  \[AB = \{x \in G | x = ab \text{ for some } a \in A , b \in B\}\]

\end{definition}

\begin{definition}[\bfseries First Isomorphism Theorem]
  Let $\phi:G \rightarrow G'$ be a surjective group homomorphism with kernel $N$. The
  quotient group $\bar{G}=G/N$ is isomorphic to the image $G'$.
  If $\pi: G \rightarrow G'$ is the canonical map, then there is a unique isomorphism
  $\bar{\phi}:\bar{G} \rightarrow G'$ s.t. $\phi = \bar{\phi} \circ \pi$:


  \begin{tikzcd}
  G \arrow[r, "\phi"] \arrow[d, "\pi"] & G' \\
  \bar{G} \arrow[ur, "\bar{\phi}"]\\
  \end{tikzcd}

\end{definition}

