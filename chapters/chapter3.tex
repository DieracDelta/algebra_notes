\section{Chapter 3}


\begin{definition}[\bfseries subspace]
  a subset $W$ of $\mathbb{R}^{n}$ is a subspace if it is not empty,
  $w_{1}\cdots, w_{n}$ are elements of $W$ and $c_{1}\cdoots c_{n}$ are scalars,
  the linear combination $c_{1}w_{1}\codts + c_{n}w_{n}$ is also in $W$.
  A subspace is \textbf{proper} if it is not trivial (${0}$) nor the entire
  $\mathbb{R}^{n}$.

\end{definition}

\begin{definition}[\bfseries nullspace]
  Subspace consisting of $\vec{x}$ that solve $AX=0$.
\end{definition}

\begin{definition}[\bfseries field]
  A field $F$ is a set together with two laws of composition,
  addition and multiplication which satisfy the following axioms:
  \begin{itemize}
    \item Addition makes F into an abelian group with identity 0
    \item multiplication makes the set of $\textbf{nonzero}$ elements
      of $F$ into an abelian group; identity is 1
    \item distributive law: $a(b+c)=ab + ac$

  \end{itemize}

  The \textbf{characteristic} is the number of times you have to add $1$ together
  to get $0$. In the field of complex numbers, the characteristic is infinite
  and therefore \textbf{characteristic zero}. In $\mathbb{F}_{p}$, the
  characteristic is $p$.

  Note that the characteristic of any field is either zero (infinite) or some
  prime, $p$. (simple proof by contradiction)


\end{definition}

\begin{definition}[\bfseries prime field]
  For some prime p, the congruence classes define a field

  \[\mathbb{F}_{p}={\bar{0},\cdots \bar{p-1}}=\mathbb{Z}/p\mathbb{Z}\]

  The multiplicative group $\mathbb{F}_{p}^{\times}$ of the prime field is a
  cyclic group of order $p-1$. (no proof provided). A generator for this cyclic
  group is called a $\textbf{primitive root}$ modulo $p$.

\end{definition}
